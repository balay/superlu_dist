% \section{Introduction}
% \label{sec:intro}

\section{About {\superlu}}

In this chapter, SuperLU will always mean Sequential SuperLU.
{\superlu} package contains a set of subroutines to solve sparse 
linear systems $AX=B$. Here $A$ is a square, nonsingular, $n\times n$
sparse matrix, and  
$X$ and $B$ are dense $n\times nrhs$ matrices, where 
$nrhs$ is the number of right-hand sides and solution vectors.
Matrix $A$ need not be symmetric or definite; indeed, {\superlu} is
particularly appropriate for matrices with very unsymmetric structure.

The package uses $LU$ decomposition with partial (or threshold) pivoting,
and forward/back substitutions. The columns of $A$ may be preordered before 
factorization (either by the user or by \superlu); 
this preordering for sparsity is completely separate from 
the factorization. To improve backward stability, we provide working 
precision iterative refinement subroutines~\cite{arioli89}. 
Routines are also available to 
equilibrate the system, estimate the condition number, calculate 
the relative backward error, and estimate error bounds for the 
refined solutions. We also include a Matlab MEX-file interface, so that
our factor and solve routines can be called as alternatives to those 
built into Matlab. The $LU$ factorization routines can handle
non-square matrices, but the triangular solves are performed only
for square matrices.

    Starting from Version 4.0, we provide the incomplete factorization
(ILU) routines which can be used as preconditioners for iterative
solvers~\cite{lishao10}.

    The factorization algorithm uses a graph reduction technique to 
reduce graph traversal time in the symbolic analysis. We exploit
dense submatrices in the numerical kernel, and organize computational
loops in a way that reduces data movement between levels of the
memory hierarchy.
The resulting algorithm is highly efficient on modern architectures.
The performance gains are particularly evident for large problems.
There are ``tuning parameters'' to optimize the peak performance as
a function of cache size. For a detailed description of the algorithm,
see reference~\cite{superlu99}.

    \superlu\ is implemented in ANSI C, and must be compiled with a standard
ANSI C compiler. It includes versions for both real and complex
matrices, in both single and double precision. 
The file names for the 
single-precision real version start with letter ``s'' (such as {\tt sgstrf.c});
the file names for the double-precision real version start with letter ``d'' 
(such as {\tt dgstrf.c}); the file names for the single-precision complex
version start with letter ``c'' (such as {\tt cgstrf.c}); the file names
for the double-precision complex version start with letter ``z'' 
(such as {\tt zgstrf.c}).

% \subsection{Availability}

\section{How to call a {\superlu} routine}
\label{sec:ex5x5}
As a simple example, let us consider how to solve
a $5\times 5$ sparse linear system $AX=B$, by calling a driver
routine {\tt dgssv()}. Figure~\ref{5x5} shows matrix $A$, and its $L$ and $U$
factors. This sample program is located in {\tt SuperLU/EXAMPLE/superlu.c.}

\begin{figure}[t]
  \newcommand{\s}{\space}
  \newcommand{\x}{\bullet}
  \newcommand{\f}{\circ}
  \begin{displaymath}
  \begin{array}{cc}
{\arraycolsep=3pt
%\normalsize
\left(
\begin{array}{rrrrr}
s  	& \s 	& u 	& u 	& \s \\
l  	& u  	& \s 	& \s 	& \s \\
\s 	& l 	& p  	& \s 	& \s \\
\s 	& \s 	& \s 	& e  	& u \\
l 	& l 	& \s 	& \s 	& r 
\end{array}
\right)  }  \:\:\: &
{\arraycolsep=3pt
%\normalsize
\left(
\begin{array}{rrrrr}
19.00  	& \s 	& 21.00 & 21.00	& \s \\
0.63 	& 21.00	& -13.26& -13.26& \s \\
\s 	& 0.57 	& 23.58	& 7.58 	& \s \\
\s 	& \s 	& \s 	& 5.00 	& 21.00 \\
0.63 	& 0.57 	& -0.24	& -0.77	& 34.20
\end{array}
\right)  }  \\ 
\: & \: \\
\mbox{Original matrix $A$} 	& \mbox{Factors $F=L+U-I$} \\
    s = 19, u = 21, p = 16, e = 5, r = 18, l = 12  & \\
  \end{array}
  \end{displaymath}
\vspace*{-0.2in}
\caption{A $5\times 5$ matrix and its $L$ and $U$ factors.}
\label{5x5}
\end{figure}

The program first initializes the three arrays,
{\tt a[], asub[]} and {\tt xa[]}, which store the nonzero coefficients of
matrix $A$, their row indices, and the indices indicating the beginning of
each column in the coefficient and row index arrays.
This storage format is called compressed column format, also known as
Harwell-Boeing format~\cite{duffgrimes92}.
% The program first reads the matrix $A$ stored 
% in the Harwell-Boeing format. See the file {\tt dreadhb.c} or
% reference~\cite{duffgrimes92} for the detailed format. In addition 
% to reading the matrix, {\tt dreadhb} allocates storage for the three arrays,
% The user can supply
% other routine to replace {\tt dreadhb}, if his or her matrix is
% stored or generated in other way. 
Next, the two utility routines
{\tt dCreate\_CompCol\_Matrix()} and {\tt dCreate\_Dense\_Matrix()}
are called to set up the matrix structures for $A$ and $B$, respectively.
The routine {\tt set\_default\_options()} sets the default values
to the input {\tt options} argument. This controls how the matrix
will be factorized and how the system will be solved.
After calling the {\superlu} routine {\tt dgssv()}, the $B$ matrix is
overwritten by the solution matrix $X$.
In the end, all the dynamically allocated data structures are de-allocated
by calling various utility routines.%

{\superlu} can perform more general tasks, which will be
explained later.

% Additional example programs are provided in Section~\ref{sec:example}.

\begin{verbatim}
#include "slu_ddefs.h"

main(int argc, char *argv[])
{
/*
 * Purpose
 * =======
 * 
 * This is the small 5x5 example used in the Sections 2 and 3 of the 
 * Users' Guide to illustrate how to call a SuperLU routine, and the
 * matrix data structures used by SuperLU.
 *
 */
    SuperMatrix A, L, U, B;
    double   *a, *rhs;
    double   s, u, p, e, r, l;
    int      *asub, *xa;
    int      *perm_r; /* row permutations from partial pivoting */
    int      *perm_c; /* column permutation vector */
    int      nrhs, info, i, m, n, nnz, permc_spec;
    superlu_options_t options;
    SuperLUStat_t stat;

    /* Initialize matrix A. */
    m = n = 5;
    nnz = 12;
    if ( !(a = doubleMalloc(nnz)) ) ABORT("Malloc fails for a[].");
    if ( !(asub = intMalloc(nnz)) ) ABORT("Malloc fails for asub[].");
    if ( !(xa = intMalloc(n+1)) ) ABORT("Malloc fails for xa[].");
    s = 19.0; u = 21.0; p = 16.0; e = 5.0; r = 18.0; l = 12.0;
    a[0] = s; a[1] = l; a[2] = l; a[3] = u; a[4] = l; a[5] = l;
    a[6] = u; a[7] = p; a[8] = u; a[9] = e; a[10]= u; a[11]= r;
    asub[0] = 0; asub[1] = 1; asub[2] = 4; asub[3] = 1;
    asub[4] = 2; asub[5] = 4; asub[6] = 0; asub[7] = 2;
    asub[8] = 0; asub[9] = 3; asub[10]= 3; asub[11]= 4;
    xa[0] = 0; xa[1] = 3; xa[2] = 6; xa[3] = 8; xa[4] = 10; xa[5] = 12;

    /* Create matrix A in the format expected by SuperLU. */
    dCreate_CompCol_Matrix(&A, m, n, nnz, a, asub, xa, SLU_NC, SLU_D, SLU_GE);
    
    /* Create right-hand side matrix B. */
    nrhs = 1;
    if ( !(rhs = doubleMalloc(m * nrhs)) ) ABORT("Malloc fails for rhs[].");
    for (i = 0; i < m; ++i) rhs[i] = 1.0;
    dCreate_Dense_Matrix(&B, m, nrhs, rhs, m, SLU_DN, SLU_D, SLU_GE);

    if ( !(perm_r = intMalloc(m)) ) ABORT("Malloc fails for perm_r[].");
    if ( !(perm_c = intMalloc(n)) ) ABORT("Malloc fails for perm_c[].");

    /* Set the default input options. */
    set_default_options(&options);
    options.ColPerm = NATURAL;

    /* Initialize the statistics variables. */
    StatInit(&stat);

    /* Solve the linear system. */
    dgssv(&options, &A, perm_c, perm_r, &L, &U, &B, &stat, &info);
    
    dPrint_CompCol_Matrix("A", &A);
    dPrint_CompCol_Matrix("U", &U);
    dPrint_SuperNode_Matrix("L", &L);
    print_int_vec("\nperm_r", m, perm_r);

    /* De-allocate storage */
    SUPERLU_FREE (rhs);
    SUPERLU_FREE (perm_r);
    SUPERLU_FREE (perm_c);
    Destroy_CompCol_Matrix(&A);
    Destroy_SuperMatrix_Store(&B);
    Destroy_SuperNode_Matrix(&L);
    Destroy_CompCol_Matrix(&U);
    StatFree(&stat);
}
\end{verbatim}


\section{Matrix data structures}
\label{sec:rep}

    \superlu\ uses a principal data structure {\tt SuperMatrix} (defined in
{\tt SRC/supermatrix.h}) to represent a general matrix, sparse or dense. 
 Figure~\ref{fig:struct} gives the specification of the {\tt SuperMatrix} 
structure. The {\tt SuperMatrix} structure contains two levels of fields.
The first level defines all the properties of a matrix which are independent
of how it is stored in memory. In particular, it specifies the following 
three orthogonal properties: storage type ({\tt Stype}) indicates the 
type of the storage scheme in {\tt *Store}; data type ({\tt Dtype}) 
encodes the four precisions; 
mathematical type ({\tt Mtype}) specifies some mathematical properties. 
The second level ({\tt *Store}) points to the actual storage
used to store the matrix. We associate with each {\tt Stype XX}
a storage format called {\tt XXformat}, such as {\tt NCformat},
{\tt SCformat}, etc.

\begin{figure}
\begin{verbatim}

typedef struct {
    Stype_t Stype; /* Storage type: indicates the storage format of *Store. */
    Dtype_t Dtype; /* Data type. */
    Mtype_t Mtype; /* Mathematical type */
    int  nrow;     /* number of rows */
    int  ncol;     /* number of columns */
    void *Store;   /* pointer to the actual storage of the matrix */
} SuperMatrix;

typedef enum {
    SLU_NC,        /* column-wise, not supernodal */
    SLU_NR,        /* row-wise, not supernodal */
    SLU_SC,        /* column-wise, supernodal */
    SLU_SR,        /* row-wise, supernodal */
    SLU_NCP,       /* column-wise, not supernodal, permuted by columns
                     (After column permutation, the consecutive columns of 
                      nonzeros may not be stored contiguously. */
    SLU_DN,        /* Fortran style column-wise storage for dense matrix */
    SLU_NR_loc     /* distributed compressed row format */ 
} Stype_t;

typedef enum {
    SLU_S,         /* single */
    SLU_D,         /* double */
    SLU_C,         /* single-complex */
    SLU_Z          /* double-complex */
} Dtype_t;

typedef enum {
    SLU_GE,        /* general */
    SLU_TRLU,      /* lower triangular, unit diagonal */
    SLU_TRUU,      /* upper triangular, unit diagonal */
    SLU_TRL,       /* lower triangular */
    SLU_TRU,       /* upper triangular */
    SLU_SYL,       /* symmetric, store lower half */
    SLU_SYU,       /* symmetric, store upper half */
    SLU_HEL,       /* Hermitian, store lower half */
    SLU_HEU        /* Hermitian, store upper half */
} Mtype_t;

\end{verbatim}
\caption{{\tt SuperMatrix} data structure.}
\label{fig:struct}
\end{figure}

The {\tt SuperMatrix} type so defined can accommodate various types 
of matrix structures and appropriate operations to be applied on them, 
although currently \superlu\ implements only a subset of this collection. 
% \SuperLU\ assumes that all matrices are stored in column-major order. 
Specifically, matrices $A$, $L$, $U$, $B$, and $X$ can have the following
types:

\begin{center}
\begin{tabular}{|l|c|c|c|c|c|} \hline
            &$A$     	  	&$L$     &$U$     &$B$     &$X$ \\\hline
{\tt Stype} &\NC\ or \NR     	&\SC     &\NC     &\DN     &\DN \\
{\tt Dtype}\footnotemark
            &any      		&any     &any     &any     &any  \\
{\tt Mtype} &\GE     		&\TRLU   &\TRU    &\GE     &\GE \\\hline
\end{tabular}
\footnotetext{{\tt Dtype} can be one of \S, \D, {\C} or \Z.}
\end{center}

In what follows, we illustrate the storage schemes defined by {\tt Stype}.
Following C's convention, all array indices and locations below are zero-based.

\begin{itemize}
\item
$A$ may have storage type {\NC} or {\NR}.
The {\NC} format is the same as the Harwell-Boeing sparse 
matrix format~\cite{duffgrimes92}, that is, the compressed column storage.
\begin{verbatim}
    typedef struct {
        int  nnz;     /* number of nonzeros in the matrix */
        void *nzval;  /* array of nonzero values packed by column */
        int  *rowind; /* array of row indices of the nonzeros */
        int  *colptr; /* colptr[j] stores the location in nzval[] and rowind[]
                         which starts column j. It has ncol+1 entries, 
                         and colptr[ncol] = nnz. */
    } NCformat;
\end{verbatim}

The {\NR} format is the compressed row storage defined below.
\begin{verbatim}
    typedef struct {
        int  nnz;     /* number of nonzeros in the matrix */
        void *nzval;  /* array of nonzero values packed by row */
        int  *colind; /* array of column indices of the nonzeros */
        int  *rowptr; /* rowptr[j] stores the location in nzval[] and colind[]
                         which starts row j. It has nrow+1 entries,
                         and rowptr[nrow] = nnz. */
    } NRformat;
\end{verbatim}

The factorization and solve routines in {\superlu} are designed to
handle column-wise storage only. If the input matrix $A$ is in row-oriented
storage, i.e., in {\NR} format, then the driver routines ({\tt dgssv()} and
{\tt dgssvx()}) actually perform the $LU$ decomposition on $A^T$, which
is column-wise, and solve the system using the $L^T$ and $U^T$ factors.
The data structures holding $L$ and $U$ on output are different (swapped)
from the data structures you get from column-wise input. For more detailed
descriptions about this process, please refer to the leading comments 
of the routines {\tt dgssv()} and {\tt dgssvx()}.
%%%% in Appendix~\ref{chap:superlu_spec}.

Alternatively, the users may call a utility routine
{\tt dCompRow\_to\_CompCol()}
to convert the input matrix in {\NR} format to another matrix
in {\NC} format, before calling SuperLU. The definition of this routine is
\begin{verbatim}
    void dCompRow_to_CompCol(int m, int n, int nnz,
                             double *a, int *colind, int *rowptr,
                             double **at, int **rowind, int **colptr);
\end{verbatim}

This conversion takes time proportional to the number of nonzeros in $A$.
However, it requires storage for a separate copy of matrix $A$.

\item
$L$ is a supernodal matrix with the storage type {\SC}.
Due to the supernodal structure, $L$ is in fact stored as a 
sparse block lower triangular matrix~\cite{superlu99}.

\begin{verbatim}
    typedef struct {
        int  nnz;           /* number of nonzeros in the matrix */
        int  nsuper;        /* index of the last supernode */
        void *nzval;        /* array of nonzero values packed by column */
        int  *nzval_colptr; /* nzval_colptr[j] stores the location in
                               nzval[] which starts column j */
        int  *rowind;       /* array of compressed row indices of 
                               rectangular supernodes */
        int  *rowind_colptr;/* rowind_colptr[j] stores the location in
                               rowind[] which starts column j */
        int  *col_to_sup;   /* col_to_sup[j] is the supernode number to 
                               which column j belongs */
        int  *sup_to_col;   /* sup_to_col[s] points to the starting column
                               of the s-th supernode */
    } SCformat;
\end{verbatim}

\item
Both $B$ and $X$ are stored as conventional two-dimensional arrays in
column-major order, with the storage type {\DN}.
\begin{verbatim}
    typedef struct {
        int lda;     /* leading dimension */
        void *nzval; /* array of size lda-by-ncol to represent 
                        a dense matrix */
    } DNformat;
\end{verbatim}
\end{itemize}

Figure~\ref{fig:matrixeg} shows the data structures for the %5\times 5$
example matrices in Figure~\ref{5x5}.

For a description of {\tt NCPformat}, see section~\ref{sec:permX}.

\begin{figure}[tbp]
\begin{itemize}
\item \begin{verbatim}
A = { Stype = SLU_NC; Dtype = SLU_D; Mtype = SLU_GE; nrow = 5; ncol = 5;
      *Store = { nnz = 12;
                 nzval = [ 19.00, 12.00, 12.00, 21.00, 12.00, 12.00, 21.00,
                           16.00, 21.00, 5.00, 21.00, 18.00 ];
                 rowind = [ 0, 1, 4, 1, 2, 4, 0, 2, 0, 3, 3, 4 ];
                 colptr = [ 0, 3, 6, 8, 10, 12 ];
               }
    }
      \end{verbatim}
\item \begin{verbatim}
U = { Stype = SLU_NC; Dtype = SLU_D; Mtype = SLU_TRU; nrow = 5; ncol = 5;
      *Store = { nnz = 11;
                 nzval = [ 21.00, -13.26, 7.58, 21.00 ];
                 rowind = [ 0, 1, 2, 0 ];
                 colptr = [ 0, 0, 0, 1, 4, 4 ];
               }
    }
      \end{verbatim}
\item \begin{verbatim}
L = { Stype = SLU_SC; Dtype = SLU_D; Mtype = SLU_TRLU; nrow = 5; ncol = 5;
      *Store = { nnz = 11;
                 nsuper = 2;
                 nzval = [ 19.00, 0.63, 0.63, 21.00, 0.57, 0.57, -13.26,
                           23.58, -0.24, 5.00, -0.77, 21.00, 34.20 ];
                 nzval_colptr = [ 0 3, 6, 9, 11, 13 ];
                 rowind = [ 0, 1, 4, 1, 2, 4, 3, 4 ];
                 rowind_colptr = [ 0, 3, 6, 6, 8, 8 ];
                 col_to_sup = [ 0, 1, 1, 2, 2 ];
                 sup_to_col = [ 0, 1, 3, 5 ];
               }
    }
      \end{verbatim}
\end{itemize}
\caption{The data structures for a $5\times 5$ matrix and its $LU$ factors, 
        as represented in the {\tt SuperMatrix} data structure.
	Zero-based indexing is used.}
  \label{fig:matrixeg}
\end{figure}



\section{{\tt Options} argument}
The {\tt options} argument is the input argument to control
the behaviour of the library. The user can tell the solver how the linear
systems should be solved based on some known characteristics of the system.
For example, for diagonally dominant matrices, 
choosing the diagonal pivots ensures stability; there is no need for
numerical pivoting (i.e., $P_r$ can be an Identity matrix).
In another situation where a sequence of matrices with the
same sparsity pattern need be factorized, the column
permutation $P_c$ (and also the row permutation $P_r$, if
the numerical values are similar) need be computed only
once, and reused thereafter.
In these cases, the solvers' performance can be much improved over
using the default settings.
{\tt Options} is implemented as a C structure containing the
following fields:
\begin{itemize}
\item {\tt Fact}\\
    Specifies whether or not the factored form of the matrix
    $A$ is supplied on entry, and if not, how the matrix $A$ will
    be factorized base on the previous history, such as factor from
    scratch, reuse $P_c$ and/or $P_r$, or reuse the data structures of $L$
    and $U$. 
    {\tt fact} can be one of:
    \begin{itemize}
    \item {\tt DOFACT}: the matrix $A$ will be factorized from scratch.
    \item {\tt SamePattern}: the matrix $A$ will be factorized assuming
	that a factorization of a matrix with the same sparsity pattern
	was performed prior to this one. Therefore, this factorization
        will reuse column permutation vector {\tt perm\_c}.
    \item {\tt SampPattern\_SameRowPerm}: the matrix $A$ will be factorized
	assuming that a factorization of a matrix with the same sparsity
	pattern and similar numerical values was performed prior to this one.
        Therefore, this factorization will reuse both row and column
        permutation vectors {\tt perm\_r} and {\tt perm\_c}, both row and
	column scaling factors $D_r$ and $D_c$, and the distributed data
	structure set up from the previous symbolic factorization.
    \item {\tt FACTORED}: the factored form of $A$ is input.
    \end{itemize}
\item {\tt Equil}  \{ {\tt YES} $|$ {\tt NO} \} \\
    Specifies whether to equilibrate the system (scale $A$'s rows and columns
    to have unit norm).
\item {\tt ColPerm}\\
    Specifies how to permute the columns of the matrix for sparsity
    preservation.
    \begin{itemize}
    \item {\tt NATURAL}: natural ordering.
    \item {\tt MMD\_ATA}: minimum degree ordering on the structure of
			$A^TA$.
    \item {\tt MMD\_AT\_PLUS\_A}: minimum degree ordering on the
			structure of $A^T+A$.
    \item {\tt COLAMD}: approximate minimum degree column ordering
    \item {\tt MY\_PERMC}: use the ordering given in {\tt perm\_c} input by
	                the user.
    \end{itemize}
\item {\tt Trans}  \{ {\tt NOTRANS} $|$ {\tt TRANS} $|$ {\tt CONJ} \} \\
    Specifies whether to solve the transposed system.
\item {\tt IterRefine} \\
    Specifies whether to perform iterative refinement, and in what
    precision to compute the residual.
    \begin{itemize}
    \item {\tt NO}: no iterative refinement
    \item {\tt SINGLE}: perform iterative refinement in single precision
    \item {\tt DOUBLE}: perform iterative refinement in double precision
    \item {\tt EXTRA}: perform iterative refinement in extra precision
    \end{itemize}
\item {\tt DiagPivotThresh}  $[0.0, 1.0]$ \\
    Specifies the threshold used for a diagonal entry to be an
    acceptable pivot.
\item {\tt SymmetricMode}  \{ {\tt YES} $|$ {\tt NO} \} \\
    Specifies whether to use the symmetric mode. Symmetric mode gives 
    preference to diagonal pivots, and uses an $(A^T + A)$-based column
    permutation algorithm.
\item {\tt PivotGrowth} \{ {\tt YES} $|$ {\tt NO} \} \\
    Specifies whether to compute the reciprocal pivot growth.
\item {\tt ConditionNumber} \{ {\tt YES} $|$ {\tt NO} \} \\
    Specifies whether to compute the reciprocal condition number.
\item {\tt RowPerm} (only for ILU or SuperLU\_DIST) \\
    Specifies whether to permute the rows of the original matrix.
    \begin{itemize}
    \item {\tt NO}: not to permute the rows
    \item {\tt LargeDiag\_MC64}: use a serial, weighted bipartite matching
      algorithm implemented in MC64 to permute the rows to make the
      diagonal large relative to the off-diagonal~\cite{duffkoster01}.
    \item {\tt LargeDiag\_AWPM}: use a parallel, approximate weighted bipartite
      matching algorithm implemented in CombBLAS to permute the rows to
      make the diagonal large relative to the off-diagonal~\cite{awpm}.
    \item {\tt MY\_PERMR}: use the permutation given by the user
    \end{itemize}
\item {\tt ILU\_DropRule} \\
    Specifies the dropping rule for ILU: ( Default: {\tt DROP\_BASIC | DROP\_AREA} )
  \begin{itemize}
  \item {\tt DROP\_BASIC:} Basic dropping rule, supernodal based ILUTP($\tau$).
  \item {\tt DROP\_PROWS:} Supernodal based ILUTP($p,\tau$), 
                   $p = \gamma\cdot nnz(A)/n$.
  \item {\tt DROP\_COLUMN:} Variant of ILUTP($p,\tau$), for j-th column,
			      $p = \gamma \cdot nnz(A(:,j))$.
  \item {\tt DROP\_AREA:}    Variation of ILUTP, for j-th column, use
		      $nnz(F(:,1:j)) / nnz(A(:,1:j))$ to control memory.
  \item {\tt DROP\_DYNAMIC:} Dynamically adjust the threshold $\tau$ during factorizaion:\\
			  If $nnz(L(:,1:j)) / nnz(A(:,1:j)) > \gamma$,
 			     $\tau_L(j) := \min(\tau_0, \tau_L(j-1)\cdot 2)$;
 			  Otherwise
			    $\tau_L(j) := \max(\tau_0, \tau_L(j-1) / 2)$. 
 			  $\tau_U(j)$ uses the similar rule.
  \item {\tt DROP\_INTERP:}  Compute the second dropping threshold by
 	                  interpolation instead of quick select (default).
   		          In this case, the actual fill ratio is not
 			  guaranteed to be smaller than gamma.
  \end{itemize}
\item {\tt ILU\_DropTol}  $[0.0, 1.0]$ \\
    Specifies the numerical dropping threshold for ILU.
\item {\tt ILU\_FillFactor} ($ \ge 1.0 $)\\
    Specifies the expected fill ratio upper bound, $\gamma$, for ILU.
\item {\tt ILU\_MILU} \{ {\tt SILU} $|$ {\tt SMILU\_1} $|$ {\tt
  SMILU\_2} $|$ {\tt SMILU\_3} \} \\
    Specifies which version of modified ILU to use.
\item {\tt PrintStat}  \{ {\tt YES} $|$ {\tt NO} \} \\
    Specifies whether to print the solver's statistics.
\end{itemize}

The routine {\tt set\_default\_options()} sets the following default
values:
\begin{verbatim}
    Fact              = DOFACT       /* factor from scratch */
    Equil             = YES             
    ColPerm           = COLAMD
    Trans             = NOTRANS
    IterRefine        = NOREFINE
    DiagPivotThresh   = 1.0          /* partial pivoting */
    SymmetricMode     = NO
    PivotGrowth       = NO;
    ConditionNumber   = NO;
    PrintStat         = YES
\end{verbatim}

To use the ILU routines, such as {\tt dgsitrf()}, the user should call
{\tt  ilu\_set\_default\_options()} to set the default values
({\tt set\_default\_options()} is first called in this routine prior
to the following):
\begin{verbatim}
    DiagPivotThresh   = 0.1          /* partial pivoting */
    RowPerm           = LargeDiag
    ILU_DropRule      = DROP_BASIC | DROP_AREA;
    ILU_DropTol       = 1e-4;
    ILU_FillFactor    = 10.0;
    ILU_Norm          = INF_NORM;
    ILU_MILU          = SILU;        /* not to use MILU */   
    ILU_FillTol       = 1e-2;
\end{verbatim}

The other possible values for each field are documented in the
source code {\tt SRC/slu\_util.h}.
The users can reset each default value according to their needs.


\section{Permutations}
\label{sec:perm}
Two permutation matrices are involved in the solution process. In fact, 
the actual factorization
we perform is $P_rAP_c^T=LU$, where $P_r$ is determined from partial pivoting 
(with a threshold pivoting option), and $P_c$ is a column permutation
chosen either by the user or {\superlu}, usually to make the $L$
and $U$ factors as sparse as possible.
$P_r$ and $P_c$ are represented by two integer vectors
{\tt perm\_r[]} and {\tt perm\_c[]}, which are the permutations
of the integers $(0:m-1)$ and $(0:n-1)$, respectively.

\subsection{Ordering for sparsity}
\label{sec:permX}
Column reordering for sparsity is completely separate from the $LU$ 
factorization. The column permutation $P_c$ should be applied before
calling the factorization routine {\tt dgstrf()}. In principle, any ordering
heuristic used for symmetric matrices can be applied to $A^TA$ 
(or $A+A^T$ if the matrix is nearly structurally symmetric) to obtain $P_c$.
Currently, we provide the following ordering options through {\tt options}
argument. The {\tt options.ColPerm} field can take the following values:
\begin{itemize}
\item {\tt NATURAL}: use natural ordefring (i.e., $P_c = I$).
\item {\tt MMD\_AT\_PLUS\_A}: use minimum degree ordering on the
			structure of $A^T+A$.
\item {\tt MMD\_ATA}: use minimum degree ordering on the structure of $A^TA$.
\item {\tt COLAMD}: use approximate minimum degree column ordering.
\item {\tt MY\_PERMC}: use the ordering given in the permutation
               vector {\tt perm\_c[]}, which is input by the user.
\end{itemize}

% The MMD code is due to Joseph W.H. Liu, which implements a variant of
% the minimum degree ordering algorithm~\cite{liu85}.

If {\tt options.ColPerm} is set to the last value, the library will use
the permutation vector {\tt perm\_c[]} as an input, which may be obtained
from any other ordering algorithm. For example, the nested-dissection type
of ordering codes include
Metis~\cite{kaku:98a}, Chaco~\cite{hele:95} and Scotch~\cite{scotch}.
% {\superlu} also contains user-callable routines to form the structure
% of $A^T + A$ or $A^TA$. These routines are named {\tt at\_plus\_a()}
% and {\tt getata()}.

Alternatively, the users can provide their own column permutation vector.
For example, it may be an ordering suitable for the underlying physical
problem. Both driver routines {\tt dgssv} and {\tt dgssvx} take 
{\tt perm\_c[]} as an input argument.
% In the future, we will augment {\tt get\_perm\_c} functionality with more
% ordering algorithms, such as approximate minimum degree ordering
% on the column intersection graph of $A$~\cite{davisgilbert00}.

After permutation $P_c$ is applied to $A$, we use {\NCP} format
to represent the permuted matrix $AP_c^T$, in which the consecutive 
columns of nonzeros may not be stored contiguously in memory.
Therefore, we need two separate arrays of pointers, {\tt colbeg[]} and 
{\tt colend[]}, to indicate the beginning and end of
each column in {\tt nzval[]} and {\tt rowind[]}.
\begin{verbatim}
    typedef struct {
        int  nnz;     /* number of nonzeros in the matrix */
        void *nzval;  /* array of nonzero values, packed by column */
        int  *rowind; /* array of row indices of the nonzeros */
        int  *colbeg; /* colbeg[j] points to the location in nzval[] and rowind[]
                         which starts column j */
        int  *colend; /* colend[j] points to one past the location in nzval[]
                         and rowind[] which ends column j */
    } NCPformat;
\end{verbatim}


\subsection{Partial pivoting with threshold}
We have included a threshold pivoting parameter $u\in [0,1]$ to control 
numerical stability. The user can choose to use a row permutation obtained
from a previous factorization. (The argument
{\tt options.Fact = SamePattern\_SameRowPerm} should be
passed to the factorization routine {\tt dgstrf()}.) 
The pivoting subroutine {\tt dpivotL()} checks whether this choice of pivot
satisfies the threshold; if not, it will try the diagonal element.
If neither of the above satisfies the threshold, 
the maximum magnitude element in the column will be used as the pivot.
The pseudo-code of the pivoting policy for column $j$ is given below.

\begin{tabbing}
junk \= junk \= junk \= \kill
\> (1)\> compute $thresh = u~|a_{mj}|$, where $|a_{mj}|=\max_{i\ge j}|a_{ij}|$;
\\ \\
\> (2)\> {\bf if} user specifies pivot row $k$ {\bf and}
                  $|a_{kj}|\ge thresh$ {\bf and} $a_{kj}\ne 0$ {\bf then} \\
\>    \> \>        pivot row $= k$; \\
\>    \>  {\bf else if } $|a_{jj}| \ge thresh$ 
                         {\bf and} $a_{jj}\ne 0$ {\bf then} \\
\>    \> \>        pivot row $= j$; \\
\>    \>  {\bf else} \\
\>    \> \>        pivot row $= m$; \\
\>    \>  {\bf endif};
\end{tabbing}

Two special values of $u$ result in the following two strategies:
\begin{itemize}
\item $u=0.0$: either use user-specified pivot order if available, 
               or else use diagonal pivot;
\item $u=1.0$: classical partial pivoting.
\end{itemize}

\section{Symmetric Mode}
In many applications, matrix $A$ may be diagonally dominant or nearly so.
In this case, pivoting on the diagonal is sufficient for stability and
is preferable for sparsity to off-diagonal pivoting.
To do this, the user can set a small (less-than-one) diagonal pivot threshold
(e.g., 0.0, 0.01) and choose an ($A^T + A$)--based column permutation
algorithm. We call this setting {\em symmetric mode}.
In this case, the {\tt options.SymmetricMode = YES} must be set.

Note that, when a diagonal entry is smaller than the
threshold, the code will still choose an off-diagonal pivot.
That is, the row permutation $P_r$ may not be Identity.
Please refer to~\cite{li05} for more discussion on the symmetric mode.


\section{Incomplete LU factorization (ILU) preconditioner}
Starting from SuperLU version 4.0, we provide the ILU routines to be
used as preconditioners for iterative solvers.  Our ILU method can be
considered to be a variant of the ILUTP method originally proposed by
Saad~\cite{saad94}, which combines a dual dropping strategy with
numerical pivoting (``T'' stands for threshold, and ``P'' stands for
pivoting).  We adapted the classic
dropping strategies of ILUTP in order to incorporate supernode
structures and to accommodate dynamic supernodes due to partial
pivoting. For the secondary dropping strategy, we proposed an
area-based fill control method, which is more flexible and numerically
robust than the traditional column-based scheme.  Furthermore, we
incorporated several heuristics for adaptively modifying various
threshold parameters as the factorization proceeds, which improves
the robustness of the algorithm. The details can be found in~\cite{lishao10}.


\section{Memory management for $L$ and $U$}
\label{sec:mem}
In the sparse $LU$ algorithm, the amount of space needed to hold the
data structures of $L$ and $U$ cannot be accurately predicted prior to 
the factorization.
The dynamically growing arrays include those for the nonzero values
({\tt nzval[]}) and the compressed row indices ({\tt rowind[]}) of $L$, and
for the nonzero values ({\tt nzval[]}) and the row
indices ({\tt rowind[]}) of $U$.

Two alternative memory models are presented to the user:
\begin{itemize}
\item system-level -- based on C's dynamic allocation capability 
      ({\tt malloc/free});
\item user-level -- based on a user-supplied {\tt work[]} array of 
      size {\tt lwork} (in bytes). This is similar to Fortran-style
      handling of work space. {\tt Work[]} is organized as a two-ended stack, 
      one end holding the $L$ and $U$ data structures, the other end 
      holding the auxiliary arrays of known size.
\end{itemize}

Except for the different ways to allocate/deallocate space, the logical 
view of the memory organization is the same for both schemes. 
Now we describe the policies in the memory module.

At the outset of the factorization, we guess there will be {\tt FILL*nnz(A)}
fills in the factors and allocate corresponding storage for the 
above four arrays, where 
{\tt nnz(A)} is the number of nonzeros in original matrix $A$, and 
{\tt FILL} is an integer, say 20. (The value of {\tt FILL} can be
set in an inquiry function {\tt sp\_ienv()}, see section~\ref{sec:parameters}.)
If this initial request exceeds the physical memory constraint, 
the {\tt FILL} factor is repeatedly reduced, and attempts are made to 
allocate smaller arrays, until the initial allocation succeeds.

During the factorization, 
if any array size exceeds the allocated bound, we expand it
as follows. We first allocate a chunk of new memory of size {\tt EXPAND}
times the old size, then copy the existing data into the new 
memory, and then free the old storage. The extra copying is necessary,
because the factorization algorithm requires that
each of the aforementioned four data structures be {\em contiguous} in memory.
The values of {\tt FILL} and {\tt EXPAND} are normally set to 20 and 1.5, 
respectively. See {\tt xmemory.c} for details.

%%%% One user suggested using ``realloc'', that can fix the problem.
After factorization, we do not garbage-collect the extra space that
may have been allocated. Thus, there will be external fragmentation in
the $L$ and $U$ data structures. The settings of {\tt FILL} and {\tt EXPAND} 
should take into account the trade-off between the number of expansions 
and the amount of fragmentation.

Arrays of known size, such as various column pointers and working arrays, 
are allocated just once. All dynamically-allocated working arrays are freed
after factorization.


\section{User-callable routines}
\label{sec:routine}

The naming conventions, calling sequences and functionality of
these routines mimic the corresponding \LAPACK\ software~\cite{lapackmanual2}.
In the routine names, such as {\tt dgstrf}, we use the two letters
{\tt GS} to denote {\em general sparse} matrices. The leading
letter{\tt x} stands for {\tt S, D, C}, or {\tt Z}, specifying the data type.

%% Appendix~\ref{chap:superlu_spec} contains, for each individual routine,
%% the leading comments and the complete specification of the calling 
%% sequence and arguments.

\subsection{Driver routines}
We provide two types of driver routines for solving systems of 
linear equations. The driver routines can handle both column-
and row-oriented storage schemes.
\begin{itemize}
\item A simple driver {\tt dgssv()}, which solves the system $AX=B$ by 
      factorizing $A$ and overwriting $B$ with the solution $X$. 
\item An expert driver {\tt dgssvx()}, which, in addition to the above, also 
      performs the following functions (some of them optionally):
      \begin{itemize}
      \item solve $A^TX=B$;
      \item equilibrate the system (scale $A$'s rows and columns to have
		unit norm) if $A$ is poorly scaled;
      \item estimate the condition number of $A$, check for near-singularity,
            and check for pivot growth;
      \item refine the solution and compute forward and backward error bounds.
      \end{itemize}
\item An expert driver {\tt dgsisx()}, which gives the approximate
  solutions of linear equations $AX=B$ or $A^TX=B$, using the ILU
  factorization from {\tt dgsitrf()}. An estimation of the condition
  number is provide, and the pivot growth is computed.
\end{itemize}

These driver routines cover all the functionality of the computational
routines. We expect that most users can simply use these driver routines
to fulfill their tasks with no need to bother with the computational routines.


\subsection{Computational routines}
The users can invoke the following computational routines, instead of the
driver routines, to directly control the behavior of {\superlu}.
The computational routines can only handle column-oriented storage.

\begin{itemize}
\item {\tt dgstrf()}: Factorize.

      This implements the first-time factorization, or later re-factorization
      with the same nonzero pattern. In re-factorizations, the code
      has the ability to use the same column permutation $P_c$ and
      row permutation $P_r$ obtained from a previous factorization.
      The input argument {\tt options} contains several scalar
      arguments to control how the $LU$ decomposition and the 
      numerical pivoting should be performed. {\tt dgstrf()} can handle
      non-square matrices.

\item {\tt dgsitrf()}: ILU.

      This implements the incomplete LU factorization
      The input argument {\tt options} contains several scalar
      arguments to control how the incomplete facotirzation and the 
      numerical pivoting should be performed.

\item {\tt dgstrs()}: Triangular solve.

      This takes the $L$ and $U$ triangular factors, the row and column 
      permutation vectors, and the right-hand side to compute a solution
      matrix $X$ of $AX=B$ or $A^TX=B$.

\item {\tt dgscon()}: Estimate condition number.
      
      Given the matrix $A$ and its factors $L$ and $U$, this estimates the 
      condition number in the one-norm or infinity-norm. The algorithm is 
      due to Hager and Higham~\cite{higham96}, and is the same as 
      {\tt CONDEST} in sparse Matlab.

\item {\tt dgsequ()/dlaqgs()}: Equilibrate.

      {\tt dgsequ} first computes the row and column scalings $D_r$ 
      and $D_c$ which would make each row and each column of the scaled 
      matrix $D_rAD_c$ have equal norm. 
      {\tt dlaqgs} then applies them to the original matrix $A$ if it is 
      indeed badly scaled. The equilibrated $A$ overwrites the original $A$.

\item {\tt dgsrfs()}: Refine solution.

      Given $A$, its factors $L$ and $U$, and an initial solution $X$, 
      this does iterative refinement, using the same precision as the 
      input data. It also computes
      forward and backward error bounds for the refined solution.

\end{itemize}

\subsection{Utility routines}
\label{sec:slu_utility}

The utility routines can help users create and destroy the {\superlu}
matrices easily. These routines reside in two places: {\tt SRC/util.c}
contains the routines that are precision-independent;\\
{\tt SRC/\{s,d,c,z\}util.c} contains the routines dependent on precision.
Here, we list the prototypes of these routines.

\begin{verbatim}
    /* Create a supermatrix in compressed column format. A is the output. */
    dCreate_CompCol_Matrix(SuperMatrix *A, int m, int n, int nnz, 
                           double *nzval, int *rowind, int *colptr,
                           Stype_t stype, Dtype_t dtype, Mtype_t mtype);

    /* Create a supermatrix in compressed row format. A is the output. */
    dCreate_CompRow_Matrix(SuperMatrix *A, int m, int n, int nnz, 
                           double *nzval, int *colind, int *rowptr,
                           Stype_t stype, Dtype_t dtype, Mtype_t mtype);

    /* Copy matrix A into matrix B, both in compressed column format. */
    dCopy_CompCol_Matrix(SuperMatrix *A, SuperMatrix *B);

    /* Create a supermatrix in dense format. X is the output.*/
    dCreate_Dense_Matrix(SuperMatrix *X, int m, int n, double *x, int ldx,
                         Stype_t stype, Dtype_t dtype, Mtype_t mtype);

    /* Create a supermatrix in supernodal format. L is the output. */
    dCreate_SuperNode_Matrix(SuperMatrix *L, int m, int n, int nnz, 
                             double *nzval, int *nzval_colptr, int *rowind,
                             int *rowind_colptr, int *col_to_sup, int *sup_to_col,
                             Stype_t stype, Dtype_t dtype, Mtype_t mtype);

    /* Convert the compressed row fromat to the compressed column format. */
    dCompRow_to_CompCol(int m, int n, int nnz, 
                        double *a, int *colind, int *rowptr,
                        double **at, int **rowind, int **colptr);

    /* Print a supermatrix in compressed column format. */
    dPrint_CompCol_Matrix(char *what, SuperMatrix *A);

    /* Print a supermatrix in supernodal format. */
    dPrint_SuperNode_Matrix(char *what, SuperMatrix *A);

    /* Print a supermatrix in dense format. */
    dPrint_Dense_Matrix(char *what, SuperMatrix *A);

    /* Deallocate the storage structure *Store. */
    Destroy_SuperMatrix_Store(SuperMatrix *A);

    /* Deallocate the supermatrix structure in compressed column format. */
    Destroy_CompCol_Matrix(SuperMatrix *A)

    /* Deallocate the supermatrix structure in supernodal format. */
    Destroy_SuperNode_Matrix(SuperMatrix *A)

    /* Deallocate the supermatrix structure in permuted compressed column format. */
    Destroy_CompCol_Permuted(SuperMatrix *A)

    /* Deallocate the supermatrix structure in dense format. */
    Destroy_Dense_Matrix(SuperMatrix *A)
\end{verbatim}

\section{Matlab interface}
\label{sec:MatlabInterface}
In the {\tt \SuperLU/MATLAB} subdirectory, we have developed a set of 
MEX-files interface to Matlab. Typing {\tt make} in this directory
produces executables to be invoked in Matlab.
The current {\tt Makefile} is set up so that the MEX-files
are compatible with Matlab Version 5. The user should edit
{\tt Makefile} for Matlab Version 4 compatibility.
Right now, only the factor routine {\tt dgstrf()} 
and the simple driver routine {\tt dgssv()} are callable by invoking
{\tt superlu} and {\tt lusolve} in Matlab, respectively. {\tt Superlu} 
and {\tt lusolve} correspond to the two Matlab built-in functions {\tt lu}
and {\tt $\backslash$}$\;$. In Matlab, when you type 

 \hspace{.4in}{\tt help superlu}

\noindent you will find the following description about {\tt superlu}'s
functionality and how to use it.
\begin{verbatim}
  SUPERLU : Supernodal LU factorization
 
  Executive summary:

  [L,U,p] = superlu(A)          is like [L,U,P] = lu(A), but faster.
  [L,U,prow,pcol] = superlu(A)  preorders the columns of A by min degree,
                                    yielding A(prow,pcol) = L*U.

  Details and options:

  With one input and two or three outputs, SUPERLU has the same effect as LU,
  except that the pivoting permutation is returned as a vector, not a matrix:

  [L,U,p] = superlu(A) returns unit lower triangular L, upper triangular U,
            and permutation vector p with A(p,:) = L*U.
  [L,U] = superlu(A) returns permuted triangular L and upper triangular U
            with A = L*U.

  With a second input, the columns of A are permuted before factoring:

  [L,U,prow] = superlu(A,psparse) returns triangular L and U and permutation 
            prow with A(prow,psparse) = L*U.
  [L,U] = superlu(A,psparse) returns permuted triangular L and triangular U 
            with A(:,psparse) = L*U.
  Here psparse will normally be a user-supplied permutation matrix or vector
  to be applied to the columns of A for sparsity.  COLMMD is one way to get
  such a permutation; see below to make SUPERLU compute it automatically.
  (If psparse is a permutation matrix, the matrix factored is A*psparse'.)

  With a fourth output, a column permutation is computed and applied:

  [L,U,prow,pcol] = superlu(A,psparse)  returns triangular L and U and
            permutations prow and pcol with A(prow,pcol) = L*U.
            Here psparse is a user-supplied column permutation for sparsity,
            and the matrix factored is A(:,psparse) (or A*psparse' if the
            input is a permutation matrix).  Output pcol is a permutation
            that first performs psparse, then postorders the etree of the 
            column intersection graph of A.  The postorder does not affect 
            sparsity, but makes supernodes in L consecutive.
  [L,U,prow,pcol] = superlu(A,0) is the same as ... = superlu(A,I); it does
            not permute for sparsity but it does postorder the etree.
  [L,U,prow,pcol] = superlu(A) is the same as ... = superlu(A,colmmd(A));
            it uses column minimum degree to permute columns for sparsity,
            then postorders the etree and factors.
\end{verbatim}


\noindent For a description about {\tt lusolve}'s functionality and how
to use it, you can type

 \hspace{.4in}{\tt help lusolve}


\begin{verbatim}
  LUSOLVE : Solve linear systems by supernodal LU factorization.
 
  x = lusolve(A, b) returns the solution to the linear system A*x = b,
      using a supernodal LU factorization that is faster than Matlab's 
      builtin LU.  This m-file just calls a mex routine to do the work.

  By default, A is preordered by column minimum degree before factorization.
  Optionally, the user can supply a desired column ordering:

  x = lusolve(A, b, pcol) uses pcol as a column permutation.  
      It still returns x = A\b, but it factors A(:,pcol) (if pcol is a 
      permutation vector) or A*Pcol (if Pcol is a permutation matrix).
       
  x = lusolve(A, b, 0) suppresses the default minimum degree ordering;
      that is, it forces the identity permutation on columns.
\end{verbatim}

% In the future, we will develop an interface for the expert driver 
% {\tt dgssvx} to provide more functionality.
Two M-files {\tt trysuperlu.m} and {\tt trylusolve.m} are written to test the 
correctness of {\tt superlu} and {\tt lusolve}. In addition to testing the
residual norms, they also test the function invocations with various
number of input/output arguments.


\section{Installation}
\label{sec:install}
\subsection{File structure}
The top level SuperLU/ directory is structured as follows:
\begin{verbatim}
    SuperLU/README    instructions on installation
    SuperLU/CBLAS/    needed BLAS routines in C, not necessarily fast
    SuperLU/EXAMPLE/  example programs
    SuperLU/INSTALL/  test machine dependent parameters; this Users' Guide
    SuperLU/MAKE_INC/ sample machine-specific make.inc files
    SuperLU/MATLAB/   Matlab mex-file interface
    SuperLU/SRC/      C source code, to be compiled into the superlu.a library
    SuperLU/TESTING/  driver routines to test correctness
    SuperLU/Makefile  top level Makefile that does installation and testing
    SuperLU/make.inc  compiler, compile flags, library definitions and C
                      preprocessor definitions, included in all Makefiles
\end{verbatim}

Before installing the package, you may need to edit {\tt SuperLU/make.inc}
for your system.
This make include file is referenced inside each of the {\tt Makefiles}
in the various subdirectories. As a result, there is no need to 
edit the {\tt Makefiles} in the subdirectories. All information that is
machine specific has been defined in {\tt make.inc}.

Sample machine-specific {\tt make.inc} are provided 
in the {\tt MAKE\_INC/} subdirectory for several systems, including
IBM RS/6000, DEC Alpha, SunOS 4.x, SunOS 5.x (Solaris), HP-PA and
SGI Iris 4.x.  When you have selected the machine on which you wish 
to install SuperLU, you may copy the appropriate sample include file 
(if one is present) into {\tt make.inc}. For example, if you wish to run 
SuperLU on an IBM RS/6000, you can do:

\hspace{.4in}{\tt cp MAKE\_INC/make.rs6k make.inc}

For systems other than those listed above, slight modifications to the 
{\tt make.inc} file will need to be made. In particular, 
the following three items should be examined:
\begin{enumerate}
\item The BLAS library.\\
   If there is a BLAS library available on your machine, you may define
   the following in {\tt make.inc}:

   \hspace{.4in}{\tt BLASDEF = -DUSE\_VENDOR\_BLAS}

   \vspace{-6pt}
   \hspace{.4in}{\tt BLASLIB = <BLAS library you wish to link with>}

   The {\tt CBLAS/} subdirectory contains the part of the C BLAS needed by 
   the {\superlu} package. However, these codes are intended for use
   only if there is no faster implementation of the BLAS already available
   on your machine. In this case, you should do the following:
   \begin{itemize}
   \item[1)] In {\tt make.inc}, undefine (comment out) BLASDEF, define:

          \hspace{.4in}{\tt BLASLIB = ../blas\$(PLAT).a}

   \item[2)] In the {\superlu/} directory, type:

          \hspace{.4in}{\tt make blaslib}

          to make the BLAS library from the routines in the {\tt CBLAS/}
	  subdirectory.
   \end{itemize}
   
\item C preprocessor definition {\tt CDEFS}.\\
   In the header file {\tt SRC/Cnames.h}, we use macros to determine how
   C routines should be named so that they are callable by Fortran.%
   \footnote{Some vendor-supplied {\BLAS} libraries do not have C 
   interfaces. So the re-naming is needed in order for the {\superlu} {\BLAS}
   calls (in C) to interface with the Fortran-style {\BLAS}.}
   The possible options for {\tt CDEFS} are:
   \begin{itemize}
   \item {\tt -DAdd\_}: Fortran expects a C routine to have an underscore
                        postfixed to the name;
   \item {\tt -DNoChange}: Fortran expects a C routine name to be identical
                        to that compiled by C;
   \item {\tt -DUpCase}: Fortran expects a C routine name to be all uppercase.
   \end{itemize}

\item The Matlab MEX-file interface.\\
   The {\tt MATLAB/} subdirectory includes Matlab C MEX-files, so that 
   our factor and solve routines can be called as alternatives to those
   built into Matlab. In the file {\tt SuperLU/make.inc}, 
   define MATLAB to be the 
   directory in which Matlab is installed on your system, for example:

   \hspace{.4in}{\tt MATLAB = /usr/local/matlab}

   At the SuperLU/ directory, type:

   \hspace{.4in} {\tt make matlabmex} 

   to build the MEX-file
   interface. After you have built the interface, you may go to the 
   {\tt MATLAB/} subdirectory to test the correctness by typing (in Matlab):
   
   \hspace{.4in}{\tt trysuperlu}

   \hspace{.4in}{\tt trylusolve}

\end{enumerate}


A {\tt Makefile} is provided in each subdirectory.
The installation can be done completely automatically by simply 
typing {\tt make} at the top level.

\subsection{Testing}
The test programs in {\tt \superlu/INSTALL} subdirectory test two routines:
\begin{itemize}
\item {\tt slamch()/dlamch()} determines properties of the floating-point
      arithmetic at run-time (both single and double precision), such as
      the machine epsilon, underflow threshold, overflow threshold, 
      and related parameters;
\item {\tt SuperLU\_timer\_()} returns the time in seconds used by the
      process. This function may need to be modified to run on your machine.
\end{itemize}

The test programs in the {\tt \superlu/TESTING} subdirectory are designed to 
test all the functions of the driver routines, especially the expert drivers.
The Unix shell script files {\tt xtest.csh} are used to invoke tests
with varying parameter settings. The input matrices include an actual 
sparse matrix {\tt \superlu/EXAMPLE/g10} of dimension $100\times 100$,%
\footnote{Matrix {\tt g10} is first generated with the structure of
the 10-by-10 five-point grid, and random numerical values. The
columns are then permuted by COLMMD ordering from Matlab.}
and numerous matrices with special properties from the \LAPACK\ test suite. 
Table~\ref{tab:testmats} describes the properties of the test matrices.

\begin{table}
\begin{minipage}{3.2in}
\begin{center}
\begin{tabular}{|l|l|} \hline
Matrix type  &Description \\\hline
0            &sparse matrix {\tt g10} \\
1            &diagonal                \\
2            &upper triangular        \\
3            &lower triangular        \\
4            &random, $\kappa=2$      \\
5            &first column zero       \\
6            &last column zero        \\
7            &last $n/2$ columns zero \\
8            &random, $\kappa=\sqrt{0.1/\varepsilon}$ \\
9            &random, $\kappa=0.1/\varepsilon$ \\
10           &scaled near underflow   \\
11           &scaled near overflow    \\ \hline
\end{tabular}
\end{center}
\caption{Properties of the test matrices. $\varepsilon$ is the machine epsilon 
         and $\kappa$ is the condition number of matrix $A$. Matrix types
         with one or more columns set to zero are used to test the error 
         return codes.}
\label{tab:testmats}
\end{minipage}
\begin{minipage}[b]{3.2in}
\begin{center}
\begin{tabular}{|l|l|l|} \hline
Test Type	&Test ratio               &Routines \\ \hline
0 &$||LU-A||/(n||A||\varepsilon)$	  &{\tt dgstrf} \\
1 &$||b-Ax|| / (||A||\;||x||\varepsilon)$ &{\tt dgssv}, {\tt dgssvx} \\
2 &$||x-x^*||/(||x^*||\kappa\varepsilon)$ &{\tt dgssvx} \\
3 &$||x-x^*|| / (||x^*||\; FERR)$    &{\tt dgssvx} \\
4 &$BERR / \varepsilon$             &{\tt dgssvx} \\ \hline
\end{tabular}
\end{center}
\caption{Types of tests. $x^*$ is the true solution, $FERR$ is the 
         error bound, and $BERR$ is the backward error.}
\label{tab:tests}
\end{minipage}
\end{table}

For each command line option specified in {\tt dtest.csh}, the test program
{\tt ddrive} reads in or generates an appropriate matrix, calls the 
driver routines, and computes a number of test ratios to verify that 
each operation has performed correctly. If the test ratio is smaller than
a preset threshold, the operation is considered to be correct.
Each test matrix is subject to the tests listed in Table~\ref{tab:tests}.

Let $r$ be the residual $r=b-Ax$, and let $m_i$ be the number of nonzeros in 
row $i$ of $A$. Then the componentwise backward error $BERR$ and 
forward error $FERR$~\cite{lapackmanual2} are calculated by:
$$ BERR = \max_i\frac{|r|_i}{(|A|~|x|+|b|)_i}\ .$$
$$FERR = \frac{||~|A^{-1}|~f~||_\infty}{||x||_\infty}\ .$$
Here, $f$ is a nonnegative vector whose components are computed as
$f_i=|r|_i + m_i~\varepsilon~(|A|~|x|+|b|)_i$, and the norm in the numerator 
is estimated using the same subroutine used for estimating the
condition number. 
$BERR$ measures the smallest relative perturbation one can make to each 
entry of A and of b so that the computed solution is an exact
solution of the perturbed problem. $FERR$ is an estimated bound on 
the error $\| x^* - x \|_{\infty} / \| x \|_{\infty}$, where $x^*$ is 
the true solution.
For further details on error analysis and error bounds
estimation, see~\cite[Chapter 4]{lapackmanual2} and ~\cite{arioli89}.


\subsection{Performance-tuning parameters}
\label{sec:parameters}
{\superlu} chooses such machine-dependent parameters as block size by calling
an inquiry function {\tt sp\_ienv()}, which may be set to return different
values on different machines. The declaration of this function is

\vspace{.1in}
{\tt int sp\_ienv(int ispec);}
\vspace{.1in}

{\tt Ispec} specifies the parameter to be returned,
(See reference~\cite{superlu99} for their definitions.)
\begin{tabbing}
xxxxxx \= xxxx \= junk \= \kill
\>ispec\>= 1: the panel size ($w$)\\
\>     \>= 2: the relaxation parameter to control supernode amalgamation 
              ($relax$)\\
\>     \>= 3: the maximum allowable size for a supernode ($maxsup$)\\
\>     \>= 4: the minimum row dimension for 2D blocking to be used ($rowblk$)\\
\>     \>= 5: the minimum column dimension for 2D blocking to be used ($colblk$)\\
\>     \>= 6: the estimated fills factor for L and U, compared with A
\end{tabbing}	    

Users are encouraged to modify this subroutine to set
the tuning parameters for their own local environment.
The optimal values depend mainly on the cache size and the \BLAS\ speed. 
If your system has a very small cache, or if you want to efficiently 
utilize the closest cache in a multilevel cache organization, you should 
pay special attention to these parameter settings. 
In our technical paper~\cite{superlu99}, we described a detailed methodology 
for setting these parameters for high performance.

The $relax$ parameter is usually set between 4 and 8. The other parameter
values which give good performance on several machines are listed in
Table~\ref{tab:block_params}. In a supernode-panel update, if the updating
supernode is too large to fit in cache, then a 2D block partitioning of
the supernode is used, in which $rowblk$ and $colblk$ determine that a 
block of size $rowblk\times colblk$ is used to update current panel.

If $colblk$ is set greater than $maxsup$, then the program will never
use 2D blocking. For example, for the Cray J90 (which does not have cache), 
$w=1$ and 1D blocking give good performance; more levels of blocking only 
increase overhead.

\begin{table}
\begin{small}
\begin{center}
\begin{tabular}{|l|r r r r r r|} \hline
\multicolumn{1}{|c|}{} &
\multicolumn{1}{r}{On-chip} &
\multicolumn{1}{r}{External} &
\multicolumn{1}{r}{} &
\multicolumn{1}{r}{} &
\multicolumn{1}{r}{} &
\multicolumn{1}{r|}{} \\
\multicolumn{1}{|c|}{Machine} &
\multicolumn{1}{r}{Cache} &
\multicolumn{1}{r}{Cache} &
\multicolumn{1}{r}{$w$} &
\multicolumn{1}{r}{$maxsup$} &
\multicolumn{1}{r}{$rowblk$} &
\multicolumn{1}{r|}{$colblk$} \\ \hline
RS/6000-590	&256 KB		&--	&8	&100 	&200	&40\\
MIPS R8000	&16 KB		&4 MB	&20	&100	&800	&100\\
Alpha 21064	&8 KB		&512 KB	&8	&100	&400	&40\\
Alpha 21164	&8 KB-L1	&4 MB	&16	&50	&100	&40\\
 		&96 KB-L2	&	&	&	&	&\\
Sparc 20	&16 KB		&1 MB	&8	&100	&400	&50\\
UltraSparc-I	&16 KB		&512 KB	&8	&100	&400	&40\\
Cray J90	&--		&--	&1	&100	&1000	&100\\
\hline
\end{tabular}
\end{center}
\end{small}
\vspace*{-.1in}
\caption{Typical blocking parameter values for several machines.}
\label{tab:block_params}
\end{table}



\section{Example programs}
\label{sec:example}

In the {\tt SuperLU/EXAMPLE/} subdirectory, we present a few sample
programs to illustrate how to use various functions provded in {\superlu}.
The users can modify these examples to suit their applications.
Here are the brief descriptions of the double precision version of
the examples:
\begin{itemize}
\item {\tt dlinsol}: use simple driver {\tt dgssv()} to solve a linear
      system one time.
\item {\tt dlinsol1}: use simple driver {\tt dgssv()} in the symmetric mode.
\item {\tt dlinsolx}: use {\tt dgssvx()} with the full (default) set of options
      to solve a linear system.
\item {\tt dlinsolx1}: use {\tt dgssvx()} to factorize $A$ first, then solve
      the system later.
\item {\tt dlinsolx2}: use {\tt dgssvx()} to solve systems repeatedly with
      the same sparsity pattern of matrix A.
\item {\tt superlu}: the small 5x5 sample program in Section~\ref{sec:ex5x5}.
\end{itemize}

In this directory, a {\tt Makefile} is provided to generate the executables,
and a {\tt README} file describes how to run these examples.


\section{Calling from Fortran}

The {\tt SuperLU/FORTRAN/} subdirectory contains an example of using
{\superlu} from a Fortran program.
The General rules for mixing Fortran and C programs are as follows.

\begin{itemize}
\item Arguments in C are passed by value, while in Fortran are passed by
      reference.  So we always pass the address (as a pointer) in the C 
      calling routine. (You cannot make a call with numbers directly in 
      the parameters.)
\item Fortran uses 1-based array addressing, while C uses 0-based.
      Therefore, the row indices ({\tt rowind[]}) and the integer pointers
      to arrays ({\tt colptr[]}) should be adjusted before they 
      are passed into a C routine.
\end{itemize}

Because of the above language differences, in order to embed {\superlu}
in a Fortran environment, users are required to use ``wrapper'' routines
(in C) for all the {\superlu} routines that will be called from Fortran
programs. The example {\tt c\_fortran\_dgssv.c} in the {\tt FORTRAN/}
directory shows how a wrapper program should be written.
This program is listed below.

\begin{verbatim}
#include "dsp_defs.h"

#define HANDLE_SIZE  8

typedef struct {
    SuperMatrix *L;
    SuperMatrix *U;
    int *perm_c;
    int *perm_r;
} factors_t;

int
c_fortran_dgssv_(int *iopt, int *n, int *nnz, int *nrhs, double *values,
                 int *rowind, int *colptr, double *b, int *ldb,
                 int factors[HANDLE_SIZE], /* a handle containing the pointer
                                              to the factored matrices */
                 int *info)

{
/* 
 * This routine can be called from Fortran.
 *
 * iopt (input) int
 *      Specifies the operation:
 *      = 1, performs LU decomposition for the first time
 *      = 2, performs triangular solve
 *      = 3, free all the storage in the end
 *
 * factors (input/output) integer array of size 8
 *      If iopt == 1, it is an output and contains the pointer pointing to
 *                    the structure of the factored matrices.
 *      Otherwise, it it an input.
 *
 */
    SuperMatrix A, AC, B;
    SuperMatrix *L, *U;
    int *perm_r; /* row permutations from partial pivoting */
    int *perm_c; /* column permutation vector */
    int *etree;  /* column elimination tree */
    SCformat *Lstore;
    NCformat *Ustore;
    int      i, panel_size, permc_spec, relax;
    trans_t  trans;
    double   drop_tol = 0.0;
    mem_usage_t   mem_usage;
    superlu_options_t options;
    SuperLUStat_t stat;
    factors_t *LUfactors;

    trans = NOTRANS;

    if ( *iopt == 1 ) { /* LU decomposition */

        /* Set the default input options. */
        set_default_options(&options);

        /* Initialize the statistics variables. */
        StatInit(&stat);

        /* Adjust to 0-based indexing */
        for (i = 0; i < *nnz; ++i) --rowind[i];
        for (i = 0; i <= *n; ++i) --colptr[i];

        dCreate_CompCol_Matrix(&A, *n, *n, *nnz, values, rowind, colptr,
                               SLU_NC, SLU_D, SLU_GE);
        L = (SuperMatrix *) SUPERLU_MALLOC( sizeof(SuperMatrix) );
        U = (SuperMatrix *) SUPERLU_MALLOC( sizeof(SuperMatrix) );
        if ( !(perm_r = intMalloc(*n)) ) ABORT("Malloc fails for perm_r[].");
        if ( !(perm_c = intMalloc(*n)) ) ABORT("Malloc fails for perm_c[].");
        if ( !(etree = intMalloc(*n)) ) ABORT("Malloc fails for etree[].");

        /*
         * Get column permutation vector perm_c[], according to permc_spec:
         *   permc_spec = 0: natural ordering 
         *   permc_spec = 1: minimum degree on structure of A'*A
         *   permc_spec = 2: minimum degree on structure of A'+A
         *   permc_spec = 3: approximate minimum degree for unsymmetric matrices
         */    	
        permc_spec = 3;
        get_perm_c(permc_spec, &A, perm_c);
	
        sp_preorder(&options, &A, perm_c, etree, &AC);

        panel_size = sp_ienv(1);
        relax = sp_ienv(2);

        dgstrf(&options, &AC, drop_tol, relax, panel_size, 
               etree, NULL, 0, perm_c, perm_r, L, U, &stat, info);

        if ( *info == 0 ) {
            Lstore = (SCformat *) L->Store;
            Ustore = (NCformat *) U->Store;
            printf("No of nonzeros in factor L = %d\n", Lstore->nnz);
            printf("No of nonzeros in factor U = %d\n", Ustore->nnz);
            printf("No of nonzeros in L+U = %d\n", Lstore->nnz + Ustore->nnz);
            dQuerySpace(L, U, &mem_usage);
            printf("L\\U MB %.3f\ttotal MB needed %.3f\texpansions %d\n",
                   mem_usage.for_lu/1e6, mem_usage.total_needed/1e6,
                   mem_usage.expansions);
        } else {
            printf("dgstrf() error returns INFO= %d\n", *info);
            if ( *info <= *n ) { /* factorization completes */
               dQuerySpace(L, U, &mem_usage);
               printf("L\\U MB %.3f\ttotal MB needed %.3f\texpansions %d\n",
                     mem_usage.for_lu/1e6, mem_usage.total_needed/1e6,
                     mem_usage.expansions);
            }
        }
	
        /* Restore to 1-based indexing */
        for (i = 0; i < *nnz; ++i) ++rowind[i];
        for (i = 0; i <= *n; ++i) ++colptr[i];

        /* Save the LU factors in the factors handle */
        LUfactors = (factors_t*) SUPERLU_MALLOC(sizeof(factors_t));
        LUfactors->L = L;
        LUfactors->U = U;
        LUfactors->perm_c = perm_c;
        LUfactors->perm_r = perm_r;
        factors[0] = (int) LUfactors;

        /* Free un-wanted storage */
        SUPERLU_FREE(etree);
        Destroy_SuperMatrix_Store(&A);
        Destroy_CompCol_Permuted(&AC);
        StatFree(&stat);

    } else if ( *iopt == 2 ) { /* Triangular solve */
        /* Initialize the statistics variables. */
        StatInit(&stat);

        /* Extract the LU factors in the factors handle */
        LUfactors = (factors_t*) factors[0];
        L = LUfactors->L;
        U = LUfactors->U;
        perm_c = LUfactors->perm_c;
        perm_r = LUfactors->perm_r;

        dCreate_Dense_Matrix(&B, *n, *nrhs, b, *ldb, SLU_DN, SLU_D, SLU_GE);

        /* Solve the system A*X=B, overwriting B with X. */
        dgstrs (trans, L, U, perm_c, perm_r, &B, &stat, info);

        Destroy_SuperMatrix_Store(&B);
        StatFree(&stat);

    } else if ( *iopt == 3 ) { /* Free storage */
        /* Free the LU factors in the factors handle */
        LUfactors = (factors_t*) factors[0];
        SUPERLU_FREE (LUfactors->perm_r);
        SUPERLU_FREE (LUfactors->perm_c);
        Destroy_SuperNode_Matrix(LUfactors->L);
        Destroy_CompCol_Matrix(LUfactors->U);
        SUPERLU_FREE (LUfactors->L);
        SUPERLU_FREE (LUfactors->U);
        SUPERLU_FREE (LUfactors);
    } else {
        fprintf(stderr, "Invalid iopt=%d passed to c_fortran_dgssv()\n");
        exit(-1);
    }
}
\end{verbatim}

Since the matrix structures in C
cannot be directly returned to Fortran, we use a handle named {\tt factors}
to access those structures. The handle is essentially
an integer pointer pointing to the factored matrices obtained from {\superlu}.
So the factored matrices are opaque objects to the Fortran program, but
can only be manipulated from the C wrapper program.

The Fortran program {\tt FORTRAN/f77\_main.f} shows how a Fortran program may
call\\
{\tt c\_fortran\_dgssv()}, and is listed below.
A {\tt README} file in this directory describes how to compile
and run this program.

\begin{verbatim}
      program f77_main
      integer maxn, maxnz
      parameter ( maxn = 10000, maxnz = 100000 )
      integer rowind(maxnz), colptr(maxn)
      real*8  values(maxnz), b(maxn)
      integer n, nnz, nrhs, ldb, info
      integer factors(8), iopt
*
*     Read the matrix file in Harwell-Boeing format
      call hbcode1(n, n, nnz, values, rowind, colptr)
*
      nrhs = 1
      ldb = n
      do i = 1, n
         b(i) = 1
      enddo
*
* First, factorize the matrix. The factors are stored in factor() handle.
      iopt = 1
      call c_fortran_dgssv( iopt, n, nnz, nrhs, values, rowind, colptr, 
     $                      b, ldb, factors, info )
*
      if (info .eq. 0) then
         write (*,*) 'Factorization succeeded'
      else
         write(*,*) 'INFO from factorization = ', info
      endif
*
* Second, solve the system using the existing factors.
      iopt = 2
      call c_fortran_dgssv( iopt, n, nnz, nrhs, values, rowind, colptr, 
     $                      b, ldb, factors, info )
*
      if (info .eq. 0) then
         write (*,*) 'Solve succeeded'
         write (*,*) (b(i), i=1, 10)
      else
         write(*,*) 'INFO from triangular solve = ', info
      endif

* Last, free the storage allocated inside SuperLU
      iopt = 3
      call c_fortran_dgssv( iopt, n, nnz, nrhs, values, rowind, colptr, 
     $                      b, ldb, factors, info )
*
      stop
      end
\end{verbatim}

% In the future, we may provide complete Fortran interfaces
% to the user-callable routines in {\superlu}.
